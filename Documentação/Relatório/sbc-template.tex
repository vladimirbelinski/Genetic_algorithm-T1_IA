\documentclass[12pt]{article}

\usepackage{sbc-template}
\usepackage{graphicx,url}
\usepackage{listings}
\usepackage{longtable}

\usepackage[brazil]{babel}   
%\usepackage[latin1]{inputenc}  
\usepackage[utf8]{inputenc}  
% UTF-8 encoding is recommended by ShareLaTex

     
\sloppy

\title{Algoritmo genético para cálculo da tabela de horários dos Componentes Curriculares do curso de Ciência da Computação da UFFS}

\author{Gabriel Batista Galli\inst{1}, Matheus Antonio Venancio Dall'Rosa\inst{1}, Vladimir Belinski\inst{1}}

\address{Ciência da Computação -- Universidade Federal da Fronteira Sul
  (UFFS)\\
  Caixa Postal 181 -- 89.802-112 -- Chapecó -- SC -- Brasil
  \email{\{g7.galli96, matheusdallrosa94, vlbelinski\}@gmail.com}
}

\begin{document} 

\maketitle
     
\begin{resumo} 
  O presente trabalho, apresentado ao curso de Ciência da Computação da Universidade Federal da Fronteira Sul - UFFS - Campus Chapecó - como requisito parcial para aprovação no Componente Curricular Inteligência Artificial, 2017.1, sob orientação do professor José Carlos Bins Filho, consiste em uma descrição detalhada da implementação de um algoritmo genético para calcular a tabela de horários dos Componentes Curriculares do curso de Ciência da Computação da UFFS.
\end{resumo}


\section{Algoritmo genético para cálculo da tabela de horários dos Componentes Curriculares do curso de Ciência da Computação da UFFS}

\subsection{Descrição geral do algoritmo}
\subsection{Representação dos genes de um indivíduo}
\subsection{Cruzamento dos indivíduos}
\subsection{Mutação dos indivíduos}
\subsection{Parâmetros utilizados na execução do algoritmo}
\subsection{Resultado final}
 

% \bibliographystyle{sbc}
% \bibliography{sbc-template}

\end{document}